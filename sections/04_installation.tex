\subsection{System requirements}

Windows intaller and Linux binaries provides all necessary components. 
To build GeoPAT 2.0 from the source code the user has to install the GDAL library and compile and install the ezGDAL and SML libraries.

\subsection{Windows installer}
The windows installer works under 64 bit versions of Windows 7, 8.1, and 10.
The installer provides four components:
\begin{itemize}
  \item{GDAL library package}
  \item{ezGDAL and SML libraries}
  \item{GeoPAT 2.0 software}
  \item{Microsoft Visual C 13 runtime libraries (optional)}
\end{itemize}
To start installation process user has to run GPAT20setup.exe.
The setup program should be ran in "Run as administrator" mode.
If an antivirus software is running on the computer, user should 
switch it off temporary for the time of GeoPAT installation.
The installer will create directory for GDAL and GeoPAT and 
copy all necessary files. Optionally, GPAT20setup.exe will
start installer of Microsoft Visual C runtime libraries.
When installation is finished, the user can find "SIL GeoPAT 2.0"
submenu in Windows start menu with "GeoPAT console" and "Uninstall GeoPAT".

The installer for Windows x64 is available at:

\begin{lstlisting}[escapechar=|]
|\url{http://sil.uc.edu/cms/data/uploads/software_data/GPAT20setup.exe}|
\end{lstlisting}

\subsection{Fedora 25 binary installation}
To install binary version of GeoPAT 2.0, user has to copy contents
of gpat20 directory from gpat20.tar.gz file to /usr/local directory.
The GeoPAT 2.0 binaries require the GDAL package installed on the computer.
The user can install this package using dnf package manager on a Fedora system.

\begin{lstlisting}
dnf install gdal
\end{lstlisting}

The additional requirement is a proper configuration of the libraries path.
The Fedora system should look for libraries in /usr/local/lib.
Sometimes it is necessary to create file local.conf containing following
text: "/usr/local/lib".
The file has to be placed in /etc/ld.so.conf.d directory.

The Fedora 25 x64 binaries are available at:

\begin{lstlisting}[escapechar=|]
|\url{http://sil.uc.edu/cms/data/uploads/software_data/gpat20.tar.gz}|
\end{lstlisting}

\subsection{Building from source code}

This compilation procedure is focused on Fedora 25
Linux distribution. The user has to modify the procedure
to fit it for a different Linux distribution.

To build GeoPAT 2.0 from the source code the user has to do four following steps:

\begin{itemize}
    \item{Install the developement version of GDAL}
    \item{Build and install the ezGDAL library}
    \item{Build and install the SML library}
    \item{Build and install the GeoPAT 2.0 software}
\end{itemize}

To install the developement version of GDAL, the dnf package manager can be used:

\begin{lstlisting}
dnf install gdal-devel
\end{lstlisting}

To install the ezGDAL library the user has to download the ezGDAL source
code and unpack it. Next, he has to compile the code by calling the following
command in an unpacked source code directory:

\begin{lstlisting}
make
\end{lstlisting}

and to install it

\begin{lstlisting}
make install
\end{lstlisting}

By default the library is placed in /usr/local/lib directory and
include file is placed in /usr/local/include.
The user can change the destination directory by adding the PREFIX parameter.

\begin{lstlisting}
make PREFIX=/my/destination/directory
\end{lstlisting}

When PREFIX is provided, the library is placed in
/my/destination/directory/lib and the include file is placed in
/my/destination/directory/include.

Installation procedure of the SML library is similar. After extraction of the source code of SML, the user should call: "make" and "make install".

The command "make install" should be called using sudo command or in
root user context.

After finishing libraries installation procedure the user has to ensure that
PREFIX/lib is on the library search path.

The last step of installation procedure is a compilation of the GeoPAT 2.0
source code. GeoPAT depends on the GDAL, SML, and ezGDAL libraries.
So, after installing above libraries the user has to unpack, compile
and install GeoPAT code.
The installation procedure is similar to installation procedures of
ezGDAL and SML libraries. PREFIX parameter works in the same way.
"make" commands should be run in the root GeoPAT 2.0 source code
directory.

The source code of GeoPAT 2.0 is available at:

\begin{lstlisting}[escapechar=|]
|\url{http://sil.uc.edu/cms/data/uploads/software_data/gpat2.0src.tar.gz}|
\end{lstlisting}

The source code of ezGDAL is available at:

\begin{lstlisting}[escapechar=|]
|\url{http://pawel.netzel.pl/data/uploads/software/libezgdal.src.tar.gz}|
\end{lstlisting}

The source code of SML is available at:

\begin{lstlisting}[escapechar=|]
|\url{http://pawel.netzel.pl/data/uploads/software/libsml.src.tar.gz}|
\end{lstlisting}